\chapter{Documentación de Base de Datos}
\label{cap:basedatos}

\section{Introducción}
Este documento describe la estructura, diseño y detalles de la base de datos utilizada en el sistema SistemaRegistroForestal. La base de datos almacena la información relacionada con zonas forestales, especies de árboles, actividades de conservación, y otros datos relevantes para la gestión y consulta de información forestal.
El objetivo principal es garantizar una gestión eficiente, segura y coherente de la información para soportar las funcionalidades del sistema.

\section{Modelo de Datos}
\subsection{Modelo Entidad-Relación (ER)}
Se presenta el diagrama ER que representa las entidades principales, sus atributos y las relaciones entre ellas.
\placeholderfigure{Modelo Entidad-Relación (ER) de la Base de Datos}

\subsection{Descripción de Entidades y Atributos}

\textbf{conservation\_activities:} Contiene información de actividades de conservación.
\begin{longtable}{|p{0.3\linewidth}|p{0.2\linewidth}|p{0.45\linewidth}|}
\hline \textbf{Campo} & \textbf{Tipo} & \textbf{Descripción} \\ \hline
\endfirsthead
\hline \textbf{Campo} & \textbf{Tipo} & \textbf{Descripción} \\ \hline
\endhead
\hline
\endfoot
\hline
\endlastfoot
id & int(11) & Identificador único de la actividad (PK, autoincremental). \\ \hline
nombre\_actividad & varchar(150) & Nombre descriptivo de la actividad. \\ \hline
fecha\_actividad & date & Fecha en la que se realizó la actividad. \\ \hline
responsable & varchar(150) & Nombre del responsable o entidad a cargo. \\ \hline
tipo\_actividad & enum & Tipo de actividad: Reforestación, Monitoreo, Control, etc. \\ \hline
descripcion & text & Descripción detallada de la actividad. \\ \hline
zona\_id & int(11) & Identificador de la zona relacionada (FK). \\ \hline
activo & boolean & Indica si la actividad está activa (1) o inactiva (0). \\ \hline
\end{longtable}

\textbf{zones:} Contiene información de zonas forestales protegidas.
\begin{longtable}{|p{0.3\linewidth}|p{0.2\linewidth}|p{0.45\linewidth}|}
\hline \textbf{Campo} & \textbf{Tipo} & \textbf{Descripción} \\ \hline
\endfirsthead
\hline \textbf{Campo} & \textbf{Tipo} & \textbf{Descripción} \\ \hline
\endhead
\hline
\endfoot
\hline
\endlastfoot
id & int(11) & Identificador único de la zona (PK, autoincremental). \\ \hline
nombre & varchar(100) & Nombre de la zona protegida. \\ \hline
ubicacion & varchar(200) & Descripción de ubicación o coordenadas geográficas. \\ \hline
provincia & varchar(100) & Provincia donde se encuentra la zona. \\ \hline
tipo\_bosque & enum & Tipo de bosque: Seco, Húmedo Tropical, Montano, Manglar, Otro \\ \hline
area\_ha & decimal(10,2) & Área total de la zona en hectáreas. \\ \hline
descripcion & text & Descripción adicional de la zona. \\ \hline
fecha\_registro & date & Fecha en que fue registrado el área. \\ \hline
\end{longtable}

\textbf{tree\_species:} Contiene información de especies arbóreas con sus características.
\begin{longtable}{|p{0.3\linewidth}|p{0.2\linewidth}|p{0.45\linewidth}|}
\hline \textbf{Campo} & \textbf{Tipo} & \textbf{Descripción} \\ \hline 
\endfirsthead
\hline \multicolumn{3}{|r|}{{Continuación de la tabla anterior}} \\
\hline \textbf{Campo} & \textbf{Tipo} & \textbf{Descripción} \\ \hline 
\endhead
\hline \multicolumn{3}{|r|}{{Continúa en la siguiente página}} \\ 
\endfoot
\hline 
\endlastfoot
id & int(11) & Identificador único de la especie (PK, autoincremental). \\ \hline
nombre\_comun & varchar(100) & Nombre común de la especie. \\ \hline
nombre\_cientifico & varchar(150) & Nombre científico. \\ \hline
familia\_botanica & varchar(100) & Familia botánica a la que pertenece. \\ \hline
estado\_conservacion & enum & Estado de conservación (ej: Vulnerable, En Peligro, etc.). \\ \hline
uso\_principal & varchar(100) & Uso principal de la especie (madera, medicina, ornamental, etc.) \\ \hline
altura\_maxima\_m & decimal(5,2) & Altura máxima aproximada en metros. \\ \hline
zona\_id & int(11) & Identificador de la zona donde se encuentra (FK). \\ \hline
\end{longtable}

\textbf{conservation\_zona:} Tabla intermedia que relaciona actividades con múltiples zonas (relación muchos a muchos).
\begin{longtable}{|p{0.3\linewidth}|p{0.2\linewidth}|p{0.45\linewidth}|}
\hline \textbf{Campo} & \textbf{Tipo} & \textbf{Descripción} \\ \hline 
\endfirsthead
\hline \multicolumn{3}{|r|}{{Continuación de la tabla anterior}} \\
\hline \textbf{Campo} & \textbf{Tipo} & \textbf{Descripción} \\ \hline 
\endhead
\hline \multicolumn{3}{|r|}{{Continúa en la siguiente página}} \\ 
\endfoot
\hline 
\endlastfoot
conservation\_id & int(11) & ID de la actividad (FK). \\ \hline
zona\_id & int(11) & ID de la zona (FK). \\ \hline
\end{longtable}

\textbf{zona\_especie:} Tabla intermedia que relaciona zonas con especies (relación muchos a muchos).
\begin{longtable}{|p{0.3\linewidth}|p{0.2\linewidth}|p{0.45\linewidth}|}
\hline \textbf{Campo} & \textbf{Tipo} & \textbf{Descripción} \\ \hline \endfirsthead
\hline \multicolumn{3}{|r|}{{Continuación de la tabla anterior}} \\
\hline \textbf{Campo} & \textbf{Tipo} & \textbf{Descripción} \\ \hline \endhead
\hline \multicolumn{3}{|r|}{{Continúa en la siguiente página}} \\ \endfoot
\hline \endlastfoot
especie\_id & int(11) & ID de la especie (FK). \\ \hline
zona\_id & int(11) & ID de la zona (FK). \\ \hline
\end{longtable}

\textbf{Relaciones:}
\begin{itemize}
    \item Una zona puede tener muchas actividades de conservación (1 a muchos).
    \item Una actividad de conservación puede estar asociada a varias zonas a través de \texttt{conservation\_zona} (muchos a muchos).
    \item Una zona puede contener muchas especies (muchos a muchos) mediante la tabla \texttt{zona\_especie}.
    \item Cada especie está asociada a una única zona en la tabla \texttt{tree\_species} pero además está relacionada con zonas en \texttt{zona\_especie} para fines más específicos.
\end{itemize}

\section{Diseño Físico}
\subsection{Motor de Base de Datos}
Para el sistema SistemaRegistroForestal, se eligió el motor de base de datos MySQL por su robustez, amplia adopción, compatibilidad con sistemas web, y soporte para operaciones transaccionales. Además, MySQL permite la definición clara de tipos de datos, relaciones, y restricciones que garantizan la integridad de la información almacenada.

\subsection{Estructura de Tablas}
\textbf{Tabla: conservation\_activities}
\begin{verbatim}
CREATE TABLE conservation_activities (
    id INT(11) AUTO_INCREMENT PRIMARY KEY,
    nombre_actividad VARCHAR(150) NOT NULL,
    fecha_actividad DATE NOT NULL,
    responsable VARCHAR(150) NOT NULL,
    tipo_actividad ENUM('Reforestación', 'Monitoreo', 
                       'Control', 'Otro') NOT NULL,
    descripcion TEXT,
    zona_id INT(11),
    activo TINYINT(1) DEFAULT 1,
    FOREIGN KEY (zona_id) REFERENCES zones(id)
);
\end{verbatim}
Descripción: Esta tabla almacena las actividades de conservación forestal realizadas en las diferentes zonas protegidas, incluyendo detalles como tipo, fecha y responsables.

\textbf{Tabla: zones}
\begin{verbatim}
CREATE TABLE zones (
    id INT(11) AUTO_INCREMENT PRIMARY KEY,
    nombre VARCHAR(100) NOT NULL,
    ubicacion VARCHAR(200),
    provincia VARCHAR(100),
    tipo_bosque ENUM('Seco', 'Húmedo Tropical', 'Montano', 
                    'Manglar', 'Otro') NOT NULL,
    area_ha DECIMAL(10,2),
    descripcion TEXT,
    fecha_registro DATE
);
\end{verbatim}
Descripción: Define las zonas o áreas protegidas donde se ejecutan actividades de conservación y habitan especies forestales.

\textbf{Tabla: tree\_species}
\begin{verbatim}
CREATE TABLE tree_species (
    id INT(11) AUTO_INCREMENT PRIMARY KEY,
    nombre_comun VARCHAR(100) NOT NULL,
    nombre_cientifico VARCHAR(150),
    familia_botanica VARCHAR(100),
    estado_conservacion ENUM('Vulnerable', 'En Peligro', 
                             'Extinto', 'Sin Riesgo') NOT NULL,
    uso_principal VARCHAR(100),
    altura_maxima_m DECIMAL(5,2),
    zona_id INT(11),
    FOREIGN KEY (zona_id) REFERENCES zones(id)
);
\end{verbatim}
Descripción: Contiene información detallada sobre las especies arbóreas que habitan en las zonas protegidas.

\textbf{Tabla: conservation\_zona}
\begin{verbatim}
CREATE TABLE conservation_zona (
    conservation_id INT(11),
    zona_id INT(11),
    PRIMARY KEY (conservation_id, zona_id),
    FOREIGN KEY (conservation_id) 
        REFERENCES conservation_activities(id),
    FOREIGN KEY (zona_id) REFERENCES zones(id)
);
\end{verbatim}
Descripción: Tabla intermedia que relaciona actividades de conservación con múltiples zonas, estableciendo una relación de muchos a muchos.

\textbf{Tabla: zona\_especie}
\begin{verbatim}
CREATE TABLE zona_especie (
    zona_id INT(11),
    especie_id INT(11),
    PRIMARY KEY (zona_id, especie_id),
    FOREIGN KEY (zona_id) REFERENCES zones(id),
    FOREIGN KEY (especie_id) REFERENCES tree_species(id)
);
\end{verbatim}
Descripción: Tabla intermedia que relaciona zonas con especies arbóreas, permitiendo una relación muchos a muchos.

\subsection{Esquema Simplificado de Tablas}
\placeholderfigure{Esquema Simplificado de Tablas}
