\chapter{Documentación de Requisitos Funcionales y No Funcionales (SRS)}
\label{cap:srs}

\section{Introducción (SRS)}
\label{srs:introduccion}

\subsection{Propósito (SRS)}
\label{srs:proposito}
Este documento tiene como objetivo definir los requisitos funcionales y no funcionales del Sistema de Registro Forestal, una aplicación destinada a registrar, consultar y actualizar información relativa a zonas forestales, especies de árboles y actividades de conservación. Será desarrollado en Java EE utilizando MySQL como sistema de gestión de base de datos, implementando una arquitectura N-Capas para asegurar la separación de responsabilidades y una alta mantenibilidad.

El documento está dirigido a los desarrolladores del sistema, stakeholders y usuarios finales del producto.

\subsection{Ámbito del sistema (SRS)}
\label{srs:ambito}
El sistema, denominado \textbf{Sistema de Registro Forestal}, permitirá a los usuarios autorizados:
\begin{itemize}
\item Registrar nuevas zonas forestales.
\item Registrar especies de árboles.
\item Registrar y consultar actividades de conservación.
\item Consultar y actualizar la información almacenada.
\end{itemize}
Este sistema está destinado a ser utilizado por entidades gubernamentales, ONGs, y organizaciones encargadas de la gestión ambiental y forestal.

Los principales beneficios que aportará el sistema son:
\begin{itemize}
\item Centralización de la información forestal
\item Facilidad de acceso y consulta de datos
\item Seguimiento de actividades de conservación
\item Mejora en la toma de decisiones relacionadas con la gestión forestal
\end{itemize}

\subsection{Definiciones, Acrónimos y Abreviaturas (SRS)}
\label{srs:definiciones}
\begin{itemize}
\item \textbf{Java EE:} Java Platform, Enterprise Edition.
\item \textbf{MySQL:} Sistema de gestión de bases de datos relacional.
\item \textbf{N-Capas:} Arquitectura de software que divide la aplicación en capas lógicas (presentación, lógica de negocio, acceso a datos).
\item \textbf{SRS:} Software Requirements Specification (Especificación de Requisitos del Software).
\item \textbf{GUI:} Interfaz gráfica de usuario.
\item \textbf{HTTPS:} Protocolo seguro de transferencia de hipertexto.
\item \textbf{MVC:} Modelo-Vista-Controlador, patrón de arquitectura de software.
\item \textbf{DAO:} Data Access Object, patrón de diseño para acceso a datos.
\end{itemize}

\subsection{Referencias (SRS)}
\label{srs:referencias}
\begin{itemize}
\item IEEE Std 830-1998: Especificaciones de los requisitos del software.
\item Java EE 8 API Documentation.
\item MySQL 8.0 Reference Manual.
\item Documentación de patrones de diseño MVC y DAO.
\end{itemize}

\subsection{Visión general del documento (SRS)}
\label{srs:vision}
El resto de este documento está organizado de la siguiente manera:
\begin{itemize}
\item \textbf{Sección 2 - Descripción general:} Proporciona una perspectiva general del sistema, sus funciones principales, características de los usuarios y restricciones.
\item \textbf{Sección 3 - Requisitos específicos:} Detalla los requisitos funcionales y no funcionales del sistema, las interfaces externas y los atributos de calidad.
\item \textbf{Sección 4 - Apéndices:} Contiene información complementaria relevante para la especificación.
\end{itemize}

\section{Descripción general del Sistema (SRS)}
\label{srs:descripcion}

\subsection{Perspectiva del producto (SRS)}
\label{srs:perspectiva}
El Sistema de Registro Forestal es una aplicación web distribuida basada en Java EE. Utiliza una arquitectura N-Capas que incluye:
\begin{itemize}
\item Capa de presentación (JSF/Servlets).
\item Capa de lógica de negocio (EJBs).
\item Capa de acceso a datos (JPA o JDBC).
\item Base de datos MySQL.
\end{itemize}
El sistema funcionará de manera independiente, aunque podrá interactuar con sistemas externos a través de interfaces para el intercambio de información geográfica o biológica si fuera necesario en el futuro.

\begin{figure}[h]
\centering
\fbox{\begin{tabular}{|c|}
\hline
\textbf{Interfaz de Usuario (Navegador Web)} \\
\hline
$\downarrow$ $\uparrow$ \\
\hline
\textbf{Capa de Presentación (JSF/Servlets)} \\
\hline
$\downarrow$ $\uparrow$ \\
\hline
\textbf{Capa de Lógica de Negocio (EJBs)} \\
\hline
$\downarrow$ $\uparrow$ \\
\hline
\textbf{Capa de Acceso a Datos (JPA/JDBC)} \\
\hline
$\downarrow$ $\uparrow$ \\
\hline
\textbf{Base de Datos MySQL} \\
\hline
\end{tabular}}
\caption{Arquitectura N-Capas del Sistema de Registro Forestal (SRS)}
\label{fig:srs_arquitectura}
\end{figure}

\subsection{Funciones del producto (SRS)}
\label{srs:funciones}
El Sistema de Registro Forestal proporcionará las siguientes funcionalidades principales:
\begin{itemize}
\item \textbf{Registro y gestión de zonas forestales:} Creación, consulta, actualización y eliminación de registros de zonas forestales con sus características geográficas y ecológicas.
\item \textbf{Gestión de especies arbóreas:} Registro y mantenimiento del catálogo de especies, asociándolas a las zonas donde se encuentran.
\item \textbf{Registro de actividades de conservación:} Documentación de intervenciones, monitoreos y acciones realizadas en cada zona forestal.
\item \textbf{Sistema de búsqueda avanzada:} Filtros por diferentes criterios (geográficos, biológicos, temporales).
\item \textbf{Generación de reportes:} Exportación de datos en formatos PDF y Excel.
\end{itemize}

\subsection{Características de los usuarios (SRS)}
\label{srs:usuarios}
El sistema contempla dos tipos principales de usuarios:
\begin{itemize}
\item \textbf{Técnico:} Profesionales del área forestal y ambiental encargados de registrar y actualizar la información. Tienen permisos para crear y modificar registros. Requieren capacitación en las funcionalidades operativas.
\item \textbf{Consultor:} Usuarios para consultar información. Pueden ser investigadores, personal de campo o público interesado. Requieren capacitación mínima sobre las funciones de consulta.
\end{itemize}

\subsection{Restricciones (SRS)}
\label{srs:restricciones}
Las siguientes restricciones afectan el desarrollo del sistema:
\begin{itemize}
\item \textbf{Tecnológicas:}
  \begin{itemize}
  \item El sistema debe ejecutarse en un servidor compatible con Java EE (GlassFish, WildFly).
  \item La base de datos debe ser MySQL 8.0 o superior.
  \item Debe ser accesible a través de los navegadores web modernos (Chrome, Firefox, Edge).
  \end{itemize}
\item \textbf{Operativas:}
  \begin{itemize}
  \item La interfaz debe estar en español.
  \item El sistema debe poder operar en redes con ancho de banda limitado.
  \end{itemize}
\item \textbf{Seguridad:}
  \begin{itemize}
  \item Debe cumplir con estándares de protección de datos personales.
  \item Las comunicaciones deben ser cifradas mediante HTTPS.
  \end{itemize}
\end{itemize}

\subsection{Suposiciones y dependencias (SRS)}
\label{srs:suposiciones}
\begin{itemize}
\item \textbf{Suposiciones:}
  \begin{itemize}
  \item Los usuarios dispondrán de acceso a internet.
  \item Los usuarios tendrán conocimientos básicos de informática.
  \item La información geográfica estará disponible en formatos estándar.
  \end{itemize}
\item \textbf{Dependencias:}
  \begin{itemize}
  \item Disponibilidad de un servidor con Java EE configurado correctamente.
  \item Disponibilidad de un servidor MySQL accesible desde el servidor de aplicaciones.
  \item Conexión a internet estable para el acceso remoto.
  \end{itemize}
\end{itemize}

\subsection{Requisitos futuros (SRS)}
\label{srs:futuros}
Los siguientes requisitos podrían considerarse para futuras versiones del sistema:
\begin{itemize}
\item Integración con sistemas de información geográfica (GIS).
\item Aplicación móvil para trabajo de campo.
\item Módulo de análisis estadístico y visualización de datos.
\item Funcionalidades de inteligencia artificial para identificación de especies.
\item Ampliación para incluir fauna asociada a cada zona forestal.
\end{itemize}

\section{Requisitos específicos (SRS)}
\label{srs:requisitos}

\subsection{Interfaces externas (SRS)}
\label{srs:interfaces}

\subsubsection{Interfaz de usuario (SRS)}
\begin{itemize}
\item El sistema proporcionará una interfaz web responsiva, compatible con resoluciones de pantalla de 1024x768 píxeles y superiores.
\item La interfaz seguirá principios de diseño adaptativo para funcionar correctamente en diferentes dispositivos.
\item Se utilizarán componentes estándar de HTML5, CSS3 y JavaScript.
\item Los formularios mostrarán validaciones instantáneas y mensajes de error claros.
\end{itemize}

\subsubsection{Interfaz de hardware (SRS)}
\begin{itemize}
\item El sistema no requiere hardware especializado para su funcionamiento básico.
\item Para funcionalidades de campo, se podría requerir integración con GPS para la versión móvil futura.
\end{itemize}

\subsubsection{Interfaz de software (SRS)}
\begin{itemize}
\item El sistema debe ser compatible con los navegadores web: Chrome (v80+), Firefox (v75+), Edge (v80+).
\item El servidor requiere Java EE 8 o superior y un servidor de aplicaciones compatible.
\item La base de datos requiere MySQL 8.0 o superior.
\end{itemize}

\subsubsection{Interfaz de comunicaciones (SRS)}
\begin{itemize}
\item Se utilizará el protocolo HTTP/HTTPS para todas las comunicaciones.
\item Las transferencias de datos utilizarán formato JSON para el intercambio de información.
\item Se implementará un mecanismo de manejo de sesiones para mantener la autenticación de los usuarios.
\end{itemize}

\subsection{Funciones (SRS)}
\label{srs:ReqFuncionales}

\subsubsection{RF01 – Registro de zonas forestales (SRS)}
\begin{itemize}
\item El sistema debe permitir registrar zonas indicando: nombre, región, coordenadas geográficas, superficie, tipo de bosque, y estado de conservación.
\item Se debe poder adjuntar archivos (imágenes, documentos) a cada registro de zona.
\item El registro debe incluir campos para metadatos adicionales (altitud, temperatura promedio, precipitación).
\end{itemize}

\subsubsection{RF02 – Consulta de zonas forestales (SRS)}
\begin{itemize}
\item El usuario podrá consultar zonas mediante filtros: nombre, región, tipo de bosque o estado.
\item Los resultados deben poder ordenarse por diferentes criterios.
\item Debe existir una vista de mapa y una vista de lista.
\item El sistema debe permitir ver el detalle completo de cada zona.
\end{itemize}

\subsubsection{RF03 – Actualización de zonas forestales (SRS)}
\begin{itemize}
\item El sistema debe permitir modificar los datos de zonas previamente registradas.
\item Los usuarios técnicos pueden realizar actualizaciones.
\end{itemize}

\subsubsection{RF04 – Gestión de especies arbóreas (SRS)}
\begin{itemize}
\item Se deben registrar especies indicando: nombre común, nombre científico, familia, altura promedio, diámetro promedio, y usos conocidos.
\item Las especies se asociarán a una o más zonas.
\item Debe permitirse adjuntar imágenes de cada especie.
\item El sistema debe facilitar la búsqueda de especies por criterios taxonómicos.
\end{itemize}

\subsubsection{RF05 – Registro y consulta de actividades de conservación (SRS)}
\begin{itemize}
\item El usuario podrá registrar actividades indicando: nombre, tipo (reforestación, monitoreo, control de plagas, etc.), fecha, responsables, y zona forestal.
\item Se podrá consultar por fecha, tipo y zona.
\item Las actividades deben poder relacionarse con las especies afectadas.
\item Debe existir un sistema de seguimiento del estado de cada actividad.
\end{itemize}

\subsubsection{RF06 – Reportes exportables (SRS)}
\begin{itemize}
\item El sistema debe permitir generar reportes en formato PDF y Excel de las zonas, especies y actividades registradas.
\item Los reportes deben ser configurables, permitiendo seleccionar qué campos incluir.
\item Se deben poder generar reportes estadísticos básicos (número de especies por zona, actividades por periodo, etc.)
\end{itemize}

\subsection{Requisitos de rendimiento (SRS)}
\label{srs:rendimiento}
\begin{itemize}
\item \textbf{RNF01 - Rendimiento:}
  \begin{itemize}
  \item Las respuestas a las consultas no deben superar los 3 segundos con hasta 1000 registros visibles.
  \item El sistema debe soportar al menos 50 usuarios concurrentes sin degradación notable del servicio.
  \item La carga inicial de la aplicación no debe superar los 5 segundos en una conexión estándar.
  \end{itemize}
\item \textbf{Carga de datos:}
  \begin{itemize}
  \item El sistema debe ser capaz de almacenar información sobre al menos 10,000 zonas forestales.
  \item Debe poder manejar un catálogo de 5,000 especies arbóreas.
  \item La base de datos debe optimizarse para manejar eficientemente consultas complejas con múltiples criterios de filtrado.
  \end{itemize}
\end{itemize}

\subsection{Restricciones de diseño (SRS)}
\label{srs:restriccionesDiseño}
\begin{itemize}
\item El desarrollo debe seguir el patrón de arquitectura MVC.
\item Se debe implementar el patrón DAO para el acceso a datos.
\item El sistema debe utilizar un sistema de inyección de dependencias.
\item La capa de presentación debe estar claramente separada de la lógica de negocio.
\item Se debe utilizar JPA como tecnología de persistencia.
\item El diseño de la base de datos debe seguir al menos la tercera forma normal.
\end{itemize}

\subsection{Atributos del sistema (SRS)}
\label{srs:atributos}

\subsubsection{Seguridad (SRS)}
\begin{itemize}
\item \textbf{RNF02 - Seguridad:}
  \begin{itemize}
  \item Las comunicaciones deben ser cifradas mediante HTTPS.
  \item Deben implementarse mecanismos contra ataques de inyección SQL y XSS.
  \item Se deben aplicar principios de desarrollo seguro para proteger la integridad de los datos.
  \end{itemize}
\end{itemize}

\subsubsection{Usabilidad (SRS)}
\begin{itemize}
\item \textbf{RNF03 - Usabilidad:}
  \begin{itemize}
  \item La interfaz debe ser intuitiva y accesible, con formularios validados y mensajes de error claros.
  \item Compatible con los navegadores modernos (Chrome, Firefox, Edge).
  \item El sistema debe ser accesible según las pautas WCAG 2.1 nivel AA.
  \item Se deben proporcionar ayudas contextuales en los formularios complejos.
  \end{itemize}
\end{itemize}

\subsubsection{Mantenibilidad y Portabilidad (SRS)}
\begin{itemize}
\item \textbf{RNF04 - Escalabilidad:}
  \begin{itemize}
  \item El sistema debe poder escalar horizontalmente para atender múltiples solicitudes simultáneas.
  \end{itemize}
\item \textbf{RNF05 - Mantenibilidad:}
  \begin{itemize}
  \item El código debe estar documentado y seguir patrones de diseño MVC y DAO.
  \item Se debe mantener un conjunto de pruebas unitarias con al menos 70\% de cobertura.
  \end{itemize}
\item \textbf{RNF06 - Disponibilidad:}
  \begin{itemize}
  \item El sistema debe estar disponible al menos el 95\% del tiempo durante el horario laboral.
  \end{itemize}
\item \textbf{RNF07 - Compatibilidad:}
  \begin{itemize}
  \item Compatible con sistemas operativos Windows, Linux y MacOS en su cliente (navegador).
  \end{itemize}
\item \textbf{RNF08 - Internacionalización:}
  \begin{itemize}
  \item Aunque inicialmente se usará en español, debe prepararse para futuras traducciones.
  \end{itemize}
\item \textbf{RNF09 - Respaldo de datos:}
  \begin{itemize}
  \item Debe realizarse respaldo automático de la base de datos cada 24 horas.
  \end{itemize}
\end{itemize}

\subsection{Otros requisitos (SRS)}
\label{srs:otros}
\begin{itemize}
\item \textbf{Cumplimiento legal:} El sistema debe cumplir con las regulaciones locales de protección de datos.
\item \textbf{Documentación:} Debe incluirse un manual de usuario completo y documentación técnica.
\item \textbf{Capacitación:} Se debe proporcionar un plan de capacitación para los diferentes tipos de usuarios.
\end{itemize}

\section{Apéndices (SRS)}
\label{srs:apendices}

\subsection{Apéndice A: Diagrama de entidades (SRS)}
Se incluirá un diagrama entidad-relación con las principales entidades:
\begin{itemize}
\item Zona Forestal
\item Especie Arbórea
\item Actividad de Conservación
\end{itemize}
\placeholderfigure{Apéndice A: Diagrama de entidades (SRS)}

\subsection{Apéndice B: Mockups de interfaz (SRS)}
Se incluirán diseños preliminares de las principales interfaces:
\begin{itemize}
\item Panel de control
\item Formulario de registro de zona forestal
\item Vista de consulta de especies
\item Formulario de actividades de conservación
\end{itemize}
\placeholderfigure{Apéndice B: Mockups de interfaz (SRS)}

\subsection{Apéndice C: Glosario de términos forestales (SRS)}
Se incluirá un glosario con terminología específica del ámbito forestal relevante para la comprensión del sistema.
